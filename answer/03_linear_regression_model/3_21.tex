\RequirePackage[l2tabu, orthodox]{nag}

\documentclass{jsarticle}
\usepackage[dvipdfmx]{graphicx}
\usepackage{physics}
\usepackage{amsmath,amssymb}

\title{3.21}

\date{\today}
\begin{document}
\maketitle

(問題)

(3.92) はエピデンスの枠組みにおける最適な$\alpha$の値である。
この結果は、次の等式を使って導出することもできる。

\begin{align*}
    \frac{\rm d}{{\rm d}\alpha}\ln|{\bf A}|={\rm Tr}\left({\bf A}^{-1}\frac{\rm d}{{\rm d}\alpha}{\bf A}\right)\tag{3.117}
    \end{align*}

実対称行列$\bf A$の固有値展開、および$\bf A$の行列式とトレースの固有値表現の標準的結果(付録 C 参照)を用いて、
この等式を証明せよ。
そして、(3.117) を用いて (3.86) から (3.92) を導け。

(参考)

\begin{align}
    {\bf A}=\alpha{\bf I}_M+\beta{\boldsymbol\Phi}^\top{\boldsymbol\Phi}\tag{3.81}
    \end{align}
\begin{align}
    E({\bf m}_N)=\frac{\beta}{2}  \left | \left | \mathbf t-\boldsymbol \Phi \mathbf m_N \right | \right | ^ 2+\frac{\alpha}{2}\mathbf m_N^\top \mathbf m_N\tag{3.82}
    \end{align}
\begin{align}
    \ln p({\bf t}|\alpha,\beta)=\frac{M}{2}\ln\alpha+\frac{N}{2}\ln\beta-E({\bf m}_N)-\frac{1}{2}\ln|{\bf A}|-\frac{N}{2}\ln(2\pi)\tag{3.86}
    \end{align}
\begin{align}
    \gamma=\sum_{i=1}^M\frac{\lambda_i}{\lambda_i+\alpha}\tag{3.91}
    \end{align}
\begin{align}
    \alpha=\frac{\gamma}{{\bf m}_N^\top{\bf m}_N}\tag{3.92}
    \end{align}
\begin{align}
    {\bf A}=\sum_{i=1}^M\lambda_i{\bf u}_i{\bf u}_i^\top\tag{C.45}
    \end{align}
\begin{align}
    {\bf A}^{-1}=\sum_{i=1}^M\frac{1}{\lambda_i}{\bf u}_i{\bf u}_i^\top\tag{C.46}
    \end{align}
\begin{align}
    |{\bf A}|=\prod_{i=1}^M\lambda_i\tag{C.47}
    \end{align}

(解答)

$\rm A$の固有値を $λ_i(i=1,\ldots,M)$とする。
また、それぞれの固有値に対応する固有ベクトルを $\bf u_i(i=1,\ldots,M)$とする。式(3.81)より、$\rm A$の固有値はパラメータ$\alpha$に依存することに注意。
なお、$\rm u_i(i=1,\ldots M)$は正規直行基底であり、
\begin{align*}
    \left\{
        \begin{array}{l}
            \bf u_i^T \bf u_j = \delta_{ij} \\
            \sum_{i=1}^M \bf u_i \bf u_i^T = \bf I
        \end{array}
        \right.
\end{align*}
となる。
式 (3.117) の左辺$\dfrac{\rm d}{{\rm d}\alpha}\ln|{\bf A}|$を計算する。

\begin{align*}
    \frac{\rm d}{{\rm d}\alpha}\ln|{\bf A}|
    &=\frac{\rm d}{{\rm d}\alpha}\ln\underbrace{\prod_{i=1}^M\lambda_i}_{(C.47)}\\
    &=\frac{\rm d}{{\rm d}\alpha}\sum_{i=1}^M\ln\lambda_i\\
    &=\sum_{i=1}^M\frac{1}{\lambda_i}\frac{\rm d\lambda_i}{{\rm d}\alpha}\tag{1}
    \end{align*}

\begin{align*}
    {\rm Tr}\left({\bf A}^{-1}\frac{\rm d}{{\rm d}\alpha}{\bf A}\right)
    &={\rm Tr}\Bigg(\underbrace{\sum_{i=1}^M\frac{1}{\lambda_i}{\bf u}_i{\bf u}_i^\top}_{(C.46)}\frac{\rm d}{{\rm d}\alpha}\sum_{j=1}^M\lambda_j{\bf u}_j{\bf u}_j^\top\Bigg)\\
    &={\rm Tr}\left(\sum_{i=1}^M\frac{1}{\lambda_i}{\bf u}_i{\bf u}_i^\top\sum_{j=1}^M\left(\frac{{\rm d}\lambda_j}{{\rm d}\alpha}{\bf u}_j{\bf u}_j^\top+\lambda_j\frac{{\rm d}{\bf u}_j}{{\rm d}\alpha}{\bf u}_j^\top+\lambda_j{\bf u}_j\frac{{\rm d}{\bf u}_j^\top}{{\rm d}\alpha}\right)\right)\\
    &={\rm Tr}\left(\sum_{i=1}^M\frac{1}{\lambda_i}{\bf u}_i{\bf u}_i^\top\sum_{j=1}^M\left(\frac{{\rm d}\lambda_j}{{\rm d}\alpha}{\bf u}_j{\bf u}_j^\top+2\lambda_j{\bf u}_j\frac{{\rm d}{\bf u}_j^\top}{{\rm d}\alpha}\right)\right)\\
    &={\rm Tr}\left(\sum_{i=1}^M\frac{1}{\lambda_i}{\bf u}_i{\bf u}_i^\top\sum_{j=1}^M\frac{{\rm d}\lambda_j}{{\rm d}\alpha}{\bf u}_j{\bf u}_j^\top+\sum_{i=1}^M\frac{1}{\lambda_i}{\bf u}_i{\bf u}_i^\top\sum_{j=1}^M2\lambda_j{\bf u}_j\frac{{\rm d}{\bf u}_j^\top}{{\rm d}\alpha}\right)\\
    &={\rm Tr}\left(\sum_{i=1}^M\sum_{j=1}^M\frac{1}{\lambda_i}\frac{{\rm d}\lambda_j}{{\rm d}\alpha}{\bf u}_i{\bf u}_i^\top{\bf u}_j{\bf u}_j^\top+\sum_{i=1}^M\sum_{j=1}^M\frac{1}{\lambda_i}2\lambda_j{\bf u}_i{\bf u}_i^\top{\bf u}_j\frac{{\rm d}{\bf u}_j^\top}{{\rm d}\alpha}\right)\\
    &={\rm Tr}\left(\sum_{i=1}^M\frac{1}{\lambda_i}\frac{{\rm d}\lambda_i}{{\rm d}\alpha}{\bf u}_i{\bf u}_i^\top+\sum_{i=1}^M2{\bf u}_i\frac{{\rm d}{\bf u}_i^\top}{{\rm d}\alpha}\right)\\
    &={\rm Tr}\left(\sum_{i=1}^M\frac{1}{\lambda_i}\frac{{\rm d}\lambda_i}{{\rm d}\alpha}{\bf u}_i{\bf u}_i^\top\right)+{\rm Tr}\left(\sum_{i=1}^M2{\bf u}_i\frac{{\rm d}{\bf u}_i^\top}{{\rm d}\alpha}\right)\\
    &=\sum_{i=1}^M\frac{1}{\lambda_i}\frac{{\rm d}\lambda_i}{{\rm d}\alpha}+{\rm Tr}\left(\sum_{i=1}^M\left(\frac{{\rm d}{\bf u}_i}{{\rm d}\alpha}{\bf u}_i^\top+{\bf u}_i\frac{{\rm d}{\bf u}_i^\top}{{\rm d}\alpha}\right)\right)\\
    &=\sum_{i=1}^M\frac{1}{\lambda_i}\frac{{\rm d}\lambda_i}{{\rm d}\alpha}+{\rm Tr}\left(\frac{{\rm d}}{{\rm d}\alpha}\sum_{i=1}^M{\bf u}_i{\bf u}_i^\top\right)\\
    &=\sum_{i=1}^M\frac{1}{\lambda_i}\frac{{\rm d}\lambda_i}{{\rm d}\alpha}+{\rm Tr}\left(\frac{{\rm d}}{{\rm d}\alpha}{\bf I}\right)\\
    &=\sum_{i=1}^M\frac{1}{\lambda_i}\frac{{\rm d}\lambda_i}{{\rm d}\alpha}\tag{2}
\end{align*}
式(1),(2) より、式 (3.117) が示された。 \\
式 (3.86) を $\alpha$ で微分して $=0$とおく。

\begin{align*}
    &\frac{{\rm d}}{{\rm d}\alpha}\ln p({\bf t}|\alpha,\beta)=0\\
    &\Leftrightarrow\frac{M}{2}\frac{{\rm d}}{{\rm d}\alpha}\ln\alpha+\frac{{\rm d}}{{\rm d}\alpha}E({\bf m}_N)-\frac{1}{2}\frac{{\rm d}}{{\rm d}\alpha}\ln|{\bf A}|=0\\
    &\Leftrightarrow\frac{M}{2\alpha}+\frac{{\rm d}}{{\rm d}\alpha}\left(\frac{\beta}{2} \left | \left | \mathbf t-\boldsymbol \Phi \mathbf m_N \right | \right | ^ 2+\frac{\alpha}{2}\mathbf m_N^\top \mathbf m_N\right)-\frac{1}{2}{\rm Tr}\left({\bf A}^{-1}\frac{\rm d}{{\rm d}\alpha}{\bf A}\right)=0\\
    &\Leftrightarrow\frac{M}{2\alpha}+\frac{1}{2}\mathbf m_N^\top \mathbf m_N-\frac{1}{2}{\rm Tr}\left({\bf A}^{-1}\right)=0\ \left(\because \frac{\rm d}{{\rm d}\alpha}{\bf A}={\bf I}\right)\\
    &\Leftrightarrow \frac{M}{2\alpha}-\frac{1}{2}{\bf m}_N^\top{\bf m}_N-\frac{1}{2}\sum_{i=1}^M\frac{1}{\lambda_i+\alpha}=0\\
    &\Leftrightarrow\alpha{\bf m}_N^\top{\bf m}_N=M-\sum_{i=1}^M\frac{\alpha}{\lambda_i+\alpha}\\
    &\Leftrightarrow\alpha{\bf m}_N^\top{\bf m}_N=\sum_{i=1}^M\frac{\lambda_i}{\lambda_i+\alpha}\\
    &\Leftrightarrow\alpha=\frac{\overbrace{\gamma}^{(3.91)}}{{\bf m}_N^\top{\bf m}_N}\tag{3}\\
    \end{align*}

式 (3) より、式 (3.92) が示された。

(問題3-23) 

対数周辺尤度関数 (3.86) の β に関する最大化が再推定方程式 (3.95) に
帰着されることを示すのにすべての段階を、 (3.86) から始めて確かめよ。

(参考)

\begin{align}
    \ln p({\bf t}|\alpha,\beta)=\frac{M}{2}\ln\alpha+\frac{N}{2}\ln\beta-E({\bf m}_N)-\frac{1}{2}\ln|{\bf A}|-\frac{N}{2}\ln(2\pi) \tag{3.86}
    \end{align}

\begin{align}
    \gamma=\sum_{i=1}^M\frac{\lambda_i}{\lambda_i+\alpha}\tag{3.91}
    \end{align}

\begin{align}
    \frac{1}{\beta}=\frac{1}{N-\gamma}\sum_{n=1}^N\left(t_n-{\bf m}_N^\top{\boldsymbol\phi}({\bf x}_n)\right)^2\tag{3.95}
    \end{align}

(解答3-23)

式(3.86)の$\beta$に関する最大化を考えるため、(3.86)を$\beta$に関して偏微分することを考える。

行列$\beta{\boldsymbol\Phi}^\top{\boldsymbol\Phi}$の固有値を$\lambda_i$とする。
行列$\beta{\boldsymbol\Phi}^\top{\boldsymbol\Phi}$を行列$\bf P$で対角化する。

\begin{align}
    {\bf P}^{-1}(\beta{\boldsymbol\Phi}^\top{\boldsymbol\Phi}){\bf P}={\rm diag}(\lambda_1,\ldots,\lambda_M)\tag{1}
    \end{align}

すると、式(3.81)から、行列$\bf A$にも対角化できる。

\begin{align}
    {\bf P}^{-1}{\bf A}{\bf P}={\rm diag}(\alpha+\lambda_1,\ldots,\alpha+\lambda_M)\tag{2}
    \end{align}

$\beta{\boldsymbol\Phi}^\top{\boldsymbol\Phi}$の固有値$\beta$に比例するため、$\lambda_i = \beta \aleph_i$とおいて、$\beta$で微分すると

\begin{align}
    \frac{d}{d\beta}\lambda_i&=\frac{d}{d\beta}\beta a_i \notag \\
    &=a_i \notag\\
    &=\frac{\lambda_i}{\beta}\tag{3}
    \end{align}

となる。
また式(2)から、$\ln |\bf A|$を$\beta$で微分すると

\begin{align}
    \frac{d}{d\beta}\ln|{\bf A}|=&\frac{d}{d\beta}\ln\prod_{i=1}^M(\lambda_i+\alpha) \notag \\
    =&\frac{d}{d\beta}\sum_{i=1}^M\ln(\lambda_i+\alpha) \notag \\
    =&\sum_{i=1}^M\frac{d}{d(\lambda_i+\alpha)}\ln(\lambda_i+\alpha)\frac{d}{d\beta}(\lambda_i+\alpha) \notag \\
    =&\frac{1}{\beta}\sum_{i=1}^M\frac{\lambda_i}{\lambda_i+\alpha} \notag \\
    =&\frac{\gamma}{\beta}\tag{4}
    \end{align}

対数周辺尤度(3.86)を$\beta$で微分して$=0$とおく。

\begin{align}
    &\frac{\partial}{\partial\beta}\ln p({\bf t}|\alpha,\beta)=0 \notag \\
    &\Leftrightarrow\frac{N}{2\beta}-\frac{1}{2}||{\bf t}-{\boldsymbol\Phi}{\bf m}_N||^2-\frac{\gamma}{2\beta}=0 \notag \\
    &\Leftrightarrow\frac{1}{\beta}=\frac{1}{N-\gamma}||{\bf t}-{\boldsymbol\Phi}{\bf m}_N||^2 \notag \\
    &\Leftrightarrow\frac{1}{\beta}=\frac{1}{N-\gamma}\sum_{n=1}^N\left(t_n-{\bf m}_N^\top{\boldsymbol\phi}({\bf x}_n)\right)^2\tag{5}
    \end{align}

式(5)より、式(3.95)が示された。

\end{document}
