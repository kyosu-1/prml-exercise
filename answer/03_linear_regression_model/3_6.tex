\RequirePackage[l2tabu, orthodox]{nag}

\documentclass{jsarticle}
\usepackage[dvipdfmx]{graphicx}

\usepackage{amsmath,amssymb}
\usepackage{physics}

\title{PRML解答集}

\date{\today}
\begin{document}
\maketitle

\begin{align*}
    p(w|t) \propto p(t|w)p(w)
\end{align*}

(3.10),(3.48)より、

\begin{align*}
    \propto \left( \prod_{n=1}^N exp\left[ -\frac{1}{2} \left\{ t_n - w^T \phi(x_n) \right\}^2 \beta \right] \right) exp\left[ -\frac{1}{2} (w - m_0)^TS_0^{-1}(w - m_0) \right] \\
    = exp\left[  -\frac{1}{2} \left\{ \beta \sum_{n=1}^N \left( t_n - w^T\phi(x_n) \right)^2 + (w - m_0)^TS_0^{-1}(w - m_0) \right\} \right]
\end{align*}

以下では exp の中の $-\frac{1}{2}$ の中身を見ていく。
ふろくの「2つの正規分布の同一性」、ここでは wの次数ごとの係数を調べていきます。そのため w で式をくくっていきます。

\begin{align*}
    = \beta \left( \begin{array}{c}
        t_1 - w^T\phi(x_1) \\
        \vdots\\
        t_N - w^T\phi(x_N)
        \end{array} \right)^T
        \left( \begin{array}{c}
        t_1 - w^T\phi(x_1) \\
        \vdots\\
        t_N - w^T\phi(x_N)
        \end{array} \right)
        + (w - m_0)^TS_0^{-1}(w - m_0)
\end{align*}


$w^T\phi(x_1) = \phi^T(x_1)w$なので

\[
    = \beta \left( \begin{array}{c}
        t_1 - \phi^T(x_1)w \\
        \vdots\\
        t_N - \phi^T(x_N)w
        \end{array} \right)^T
        \left( \begin{array}{c}
        t_1 - \phi^T(x_1)w \\
        \vdots\\
        t_N - \phi^T(x_N)w
        \end{array} \right)
        + (w - m_0)^TS_0^{-1}(w - m_0)
    \]

ここで、
\[
    \Phi = \left( \begin{array}{c}
        \phi^T(x_1) \\
        \vdots\\
        \phi^T(x_N)
        \end{array} \right)
    \]

であるので、定数項を C とおくとして
\begin{align*}
    = \beta (t - \Phi w)^T(t - \Phi w) + (w - m_0)^TS_0^{-1}(w - m_0)\\
    = \beta ( w^T\Phi^T\Phi w - w^T\Phi^Tt - t^T\Phi w ) + w^TS_0^{-1}w - w^TS_0^{-1}m_0 - m_0^TS_0^{-1}w + C
\end{align*}

$S_0$ は共分散なので対称行列となる。よってその逆行列も対称行列なので $(S_0^{-1})^T = S_0^{-1}$。

\[
    = w^T(S_0^{-1} + \beta \Phi^T\Phi)w - w^T(S_0^{-1}m_0 + \beta \Phi^T t) - (S_0^{-1}m_0 + \beta \Phi^T t)^Tw + C
\]

ここで $R = S_0^{-1} m_0 + \beta \Phi^Tt$ とすると

(3.49) を書き下して上記の式と一致することを確認する。

\begin{align*}
    N(w|m_N, S_N)
    \propto exp\left\{ -\frac{1}{2} (w - m_N)^T S_N^{-1} (w - m_N) \right\}
\end{align*}

ここでも exp の中の $-\frac{1}{2}$ の中のみを見ていきます。

\[
    (w - m_N)^T S_N^{-1} (w - m_N)
    \]

ここで、 $m_N, S_N$ はそれぞれ以下のように与えられています。

\begin{align*}
    m_N = S_N(S_0^{-1} m_0 + \beta \Phi^Tt) = S_NR\\
    S_N^{-1} = S_0^{-1} + \beta \Phi^T \Phi
\end{align*}

ここで $S_N$ は事後分布から導き出した目標の式にも出てきますので展開せず、 $m_N$ を展開していきます。

\begin{align*}
    (w - m_N)^T S_N^{-1} (w - m_N)\\
    = (w - S_NR)^T S_N^{-1} (w - S_NR)\\
    = w^T S_N^{-1} w - w^T R - R^Tw + C
\end{align*}

よって、事後分布 $p(w|t)$ と $N(w|m_N, S_N)$ は同じ正規分布となります。
\end{document}
