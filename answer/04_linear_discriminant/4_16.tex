\RequirePackage[l2tabu, orthodox]{nag}

\documentclass{jsarticle}
\usepackage[dvipdfmx]{graphicx}

\usepackage{amsmath,amssymb}
\usepackage{physics}

\title{PRML4.16}

\date{\today}
\begin{document}
\maketitle

(問題)\\
$t= 0$または$t = 1$に対応する2クラスの1つに属することが知られている各観測値$x_n$
における2値分類問題を考える。このとき、学習データがときどき間違ったラベルを付けられるため、学習データの収集手順は完全なものではないと仮定する。
すべてのデータ$x_n$に対し、クラスラベルの値$t_n$を与える代わりに、$t_n=1$となる確率を表現する値$π_n$を与える。
確率モデル$p(t=1|\phi)$が与えられた場合、そのようなデータ集合に適切な対数尤度関数を記述せよ。

(解答) \\
クラスラベル$t_n$の代わりに$t_n=1$となる確率$\pi_n$が与えられているので、
確率モデル$p(t=1|\phi)$のもとで$t_n$が生起される確率は、
\begin{align*}
    p(t_n) = p(t_n=1|\phi_n)^{\pi_n} \{ 1 - p(t_n=1|\phi_n) \}^{1-\pi_n}
\end{align*}
よってデータ集合に適切な対数尤度関数は

\begin {align*}
\ln p\left(\mathbf{t} | \phi\right)=\sum_{n=1}^N \{\pi_{n} \ln p(t_n=1|\phi_n)+(1-\pi_{n} ) \ln \left(1-p(t_n=1|\phi_n)\right)\}
\end {align*}

\end{document}
