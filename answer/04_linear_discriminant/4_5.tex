\RequirePackage[l2tabu, orthodox]{nag}

\documentclass{jsarticle}
\usepackage[dvipdfmx]{graphicx}

\usepackage{amsmath,amssymb}
\usepackage{physics}

\title{PRML4.5}

\date{\today}
\begin{document}
\maketitle

(問題)

(4.20),(4.23)と(4.24)を使って、フィッシャーの判別規準(4.25)が(4.26)の形で書けることを示せ。
\\ 

(解答)

\begin{align*}
    y={\bf w}^\top{\bf x} \tag{4.20}
\end{align*}
\begin{align*}
    m_k={\bf w}^\top{\bf m}_k\tag{4.23}
    \end{align*}
\begin{align*}
    s_k^2 = \sum_{n \in C_k} (y_n - m_k)^2 \tag{4.24}
\end{align*}
\begin{align*}
    J(\bf w) = \cfrac{(m_2 - m_1)^2}{s_1^2 + s_2^2} \tag{4.25}
\end{align*}
\begin{align*}
    J({\bf w})=\frac{{\bf w}^\top{\bf S}_{\rm B}{\bf w}}{{\bf w}^\top{\bf S}_{\rm W}{\bf w}}\tag{4.26}
    \end{align*}
\begin{align*}
    {\bf S}_{\rm B}=({\bf m}_2-{\bf m}_1)({\bf m}_2-{\bf m}_1)^\top\tag{4.27}
    \end{align*}
\begin{align*}
    {\bf S}_{\rm W}=\sum_{n\in{\mathcal C}_1}({\bf x}-{\bf m}_1)({\bf x}-{\bf m}_1)^\top+\sum_{n\in{\mathcal C}_2}({\bf x}-{\bf m}_2)({\bf x}-{\bf m}_2)^\top\tag{4.28}
    \end{align*}

式(4.23),(4.24)より
\begin{align*}
    s_k^2
    &=\sum_{n\in{\mathcal C}_k}({\bf w}^\top{\bf x}_n-{\bf w}^\top{\bf m}_k)^2\\
    &=\sum_{n\in{\mathcal C}_k}({\bf w}^\top({\bf x}_n-{\bf m}_k))^2\\
    &=\sum_{n\in{\mathcal C}_k}( ({\bf x}_n-{\bf m}_k) ^\top{\bf w})^2\\
    &=\sum_{n\in{\mathcal C}_k}( ({\bf x}_n-{\bf m}_k) ^\top{\bf w})^\top( ({\bf x}_n-{\bf m}_1) ^\top{\bf w})\\
    &=\sum_{n\in{\mathcal C}_k}{\bf w}^\top({\bf x}_n-{\bf m}_k)({\bf x}_n-{\bf m}_k) ^\top{\bf w}
    \end{align*}
以上の式より
\begin{align*}
    J({\bf w})
    &=\frac{({\bf w}^\top{\bf m}_2-{\bf w}^\top{\bf m}_1)^2}{\displaystyle\sum_{n\in{\mathcal C}_1}{\bf w}^\top({\bf x}_n-{\bf m}_1)({\bf x}_n-{\bf m}_1) ^\top{\bf w}+\displaystyle\sum_{n\in{\mathcal C}_2}{\bf w}^\top({\bf x}_n-{\bf m}_2)({\bf x}_n-{\bf m}_2) ^\top{\bf w}}\\
    &=\frac{{\bf w}^\top({\bf m}_2-{\bf m}_1)({\bf m}_2-{\bf m}_1){\bf w}}{{\bf w}^\top\left(\displaystyle\sum_{n\in{\mathcal C}_1}({\bf x}_n-{\bf m}_1)({\bf x}_n-{\bf m}_1) ^\top+\displaystyle\sum_{n\in{\mathcal C}_2}({\bf x}_n-{\bf m}_2)({\bf x}_n-{\bf m}_2) ^\top\right){\bf w}}\\
    &=\frac{{\bf w}^\top{\bf S}_{\rm B}{\bf w}}{{\bf w}^\top{\bf S}_{\rm W}{\bf w}}
    \end{align*}

(補足)
フィッシャーの線形判別とはクラス間分散とクラス内分散の比を最大にするような$\bf w$で射影し、入力データの次元を減らすような特徴量抽出の手法である。
あくまで特徴量抽出の手法であって、これ自体が識別を行うものではないが、しきい値を決めることで識別を行うことが出来る。 \\
\end{document}
